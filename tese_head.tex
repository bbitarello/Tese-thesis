\usepackage[english,brazilian]{babel} % suporte para línguas
\usepackage[utf8]{inputenc} % codificação


\usepackage{subfig, epsfig}
\captionsetup[subfigure]{style=default, 
  margin=0pt, parskip=0pt, hangindent=0pt, indention=0pt, 
  singlelinecheck=true, labelformat=parens, labelsep=space}

\usepackage{ae}
\usepackage{aecompl}
\usepackage{booktabs}
\usepackage[T1] {fontenc}
\usepackage{lmodern} %test

\usepackage{footnote}
\usepackage{geometry}
\geometry{includeheadfoot,verbose,tmargin=2.7cm,bmargin=2.9cm}
%originally bmargin=3.7 and tmargin=2.8
\usepackage{fancyhdr}
\usepackage{afterpage}

\usepackage[table,xcdraw]{xcolor}

\usepackage{multirow}
\usepackage{csquotes}
\usepackage{placeins}


% Notas criadas nas tabelas ficam no fim das tabelas
\makesavenoteenv{tabular}
\usepackage{longtable}

\pagestyle{fancy}

%\fancypagestyle{plain}
%{
%  \fancyhf{}%
%  \renewcommand{\headrulewidth}{0pt}%
%  \fancyfoot[C]{\thepage}
%}

\usepackage{xr}
\usepackage{graphicx,wrapfig} % para incluir figuras
\usepackage[all]{xy} % para incluir diagramas
\usepackage{amsfonts, amssymb, amsthm, amsmath, amscd, textcomp} % pacote AMS

\usepackage{color, float, bbm, multicol, rotating}
\usepackage{verbatim, listings, booktabs}


\usepackage{caption} % Customizar as legendas de figuras e tabelas
\usepackage{array} % Elementos extras para formatação de tabelas

\usepackage[switch*,pagewise]{lineno} % números nas linhas

\renewcommand{\thefigure}{\arabic{figure}}
\renewcommand{\thetable}{\arabic{table}}

\extrafloats{100}
\usepackage[maxfloats=50]{morefloats}

\usepackage {tocvsec2} % controlar profundidade de table of contents
\setcounter {secnumdepth}{0}
\setcounter {tocdepth}{2}

%\widowpenalty10000
%\clubpenalty10000

\usepackage{mathpazo} % fonte palatino
\usepackage{hyperref}
\usepackage[numbered]{bookmark} %test

% Adicionar bibliografia, índice e conteúdo na Tabela de conteúdo
% Não inclui lista de tabelas e figuras no índice
\usepackage[nottoc,notlof,notlot, notindex]{tocbibind}

\usepackage{icomma} % Posicionar inteligentemente a vírgula como separador decimal
\usepackage[tight]{units} % Formatar as unidades com as distâncias corretas

\usepackage{setspace}

\usepackage{lastpage} % Conta o número de páginas
%\usepackage{pageslts} %test
\usepackage{pdflscape} % ambiente landscape

%\usepackage[round]{natbib}
 
%\usepackage[sectionbib, round]{natbib}
%\usepackage{chapterbib}
%\bibliographystyle{disertacao}


\usepackage[firstinits=true,
			uniquename=false,
            uniquelist=false,
            sorting=anyt,
			url=true,
            doi=false,
            isbn=true,
            maxcitenames=2,
            mincitenames=1,
            maxbibnames=10,
            bibencoding=inputenc,
            hyperref=auto,
            style=authoryear,            
            refsection=chapter]
{biblatex}
%add parenthesis in volume/issue in references
\renewbibmacro*{volume+number+eid}{%
  \printfield{volume}%
%  \setunit*{\adddot}% DELETED
  \setunit*{\addnbspace}% NEW (optional); there's also \addnbthinspace
  \printfield{number}%
  \setunit{\addcomma\space}%
  \printfield{eid}}
\DeclareFieldFormat[article]{number}{\mkbibparens{#1}}

%add comma between author and year
\renewcommand*{\nameyeardelim}{\addcomma\space}

\AtEveryBibitem{\ifentrytype{article}{\clearfield{issn}}{}} %do not print ISSN for articles

\AtEveryBibitem{\ifentrytype{article}{\clearfield{url}}{}} %do not print URL unless it is a thesis.

\bibliography{Thesis}



\usepackage[flushleft]{threeparttable}
\usepackage{tabularx}

\usepackage{pdfpages}

\usepackage{fixme}
%\usepackage{pdfcomment}

\fxsetup{layout={footnote}}

\usepackage{titlesec}
\usepackage{etoolbox}
\usepackage{chngcntr}
\counterwithout{equation}{chapter}
\usepackage{lineno}

\usepackage{lettrine} %fancy big letter

\usepackage{epstopdf}
\epstopdfDeclareGraphicsRule{.tiff}{png}{.png}{convert #1 \OutputFile}
\AppendGraphicsExtensions{.tiff}  %convert tiff images tp PNG

% --- definições gerais ---
\newcommand{\barra}{\backslash}
\newcommand{\To}{\longrightarrow}
\newcommand{\abs}[1]{\left\vert#1\right\vert}
\newcommand{\set}[1]{\left\{#1\right\}}
\newcommand{\seq}[1]{\left<#1\right>}
\newcommand{\norma}[1]{\left\Vert#1\right\Vert}
\newcommand{\hr}{\par\noindent\hrulefill\par}
\definecolor{airforceblue}{rgb}{0.36, 0.54, 0.66}
%\preto\chapter{\color{airforceblue}}
%\titleformat{\chapter}[hang]{\Huge\bfseries}{\thechapter\hsp\textcolor{gray75}{|}\hsp}{0pt}{\Huge\bfseries}
% --- ---


%\newcommand{\titulo}{Má-adaptação como subproduto da adaptação: um estudo em escala genômica}
%\newcommand{\nomedoaluno}{Bárbara Domingues Bitarello}
%\newcommand{\advisor}{Diogo Meyer} \newcommand{\ano}{2015}
%\hypersetup{colorlinks=true, linkcolor=black, citecolor=black,
%  filecolor=black, pagecolor=black, urlcolor=black,
%  pdfauthor={\nomedoaluno}, pdftitle={\titulo}, pdfsubject={Genética e
%    Biologia Evolutiva}, pdfkeywords={Genética Quantitiva, Morfologia
%    Craniana, Roedores}, pdfproducer={Latex}, pdfcreator={pdflatex}}
 
%\geometry{bindingoffset=20pt}
%\geometry{paperwidth=290mm, paperheight=297mm, margin=1in}
%\setlength{\marginparwidth=80mm}


